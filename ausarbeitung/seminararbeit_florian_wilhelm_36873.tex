%
% seminararbeit_florian_wilhelm_36873 Florian Wilhelm
% Erstellt am 19.03.2014
%
% Bauen mit dem beiliegenden Makefile

\documentclass[11pt,a4paper,titlepage,ngerman]{scrartcl}

% Deutsch mit neuer Rechtschreibung
\usepackage[ngerman]{babel}

% Umlaute direkt eingeben http://www.jr-x.de/publikationen/latex/tipps/besonderheiten.html
% Hatte hier ursprünglich "latin1" als Option, das ging aber nicht. Hängt wohl von der Codierung des Dokumentes ab.
\usepackage[utf8]{inputenc} 

% Einbinden von Grafiken
\usepackage{graphicx}

\usepackage{listings}

% Setzen der Eigenschaften der erzeugten PDF-Datei
\usepackage[pdftex,
    pdfauthor={Florian Wilhelm},
    pdfsubject={Seminararbeit Florian Wilhelm MatrNr 36873},
    pdftitle={Futures and Promises in Scala},    
    pdfproducer={Latex with hyperref},
    pdfcreator={pdflatex}]{hyperref}

% Glossar-Paket laden und Glossar erzeugen
% Hierfür sind mehrere Übersetzungsvorgänge nötig. Dafür gibt es ein Build-Target im Makefile.
\usepackage[toc]{glossaries}
\makeglossaries

%
% Glossar
%
\newglossaryentry{sip}{name={Scala Improvement Process}, description={ist eine Platform auf der Verbesserungen der Programmiersprache Scala vorgestellt werden. Es ist notwendig ein SIP-Dokument einzureichen, wenn man die Sprache oder die Standardbibliothek erweitern möchte. Die Plattform ist unter \url{http://docs.scala-lang.org/sips/} zu erreichen}}

% Template: \newglossaryentry{yyy}{name=xxx, description={}} 

%
% Glossar Ende
%

%Römische Zahlen
% http://www.mrunix.de/forums/showthread.php?t=44246
\newcommand{\RM}[1]{\MakeUppercase{\romannumeral #1{.}}}

\begin{document}

% Zeilenumbrüche in Code-listings
\lstset{breaklines=true}

\titlehead{
	\includegraphics[width=0.9\linewidth]{pic/500px-Hska_logo.png}
}
\title{Seminararbeit}
\subtitle{Futures and Promises in Scala}
\author{Florian Wilhelm \\
		Matrikelnummer: 36873\\
		Sommersemester 2014}
\publishers{
    \textbf{Betreuer:} Prof. Dr. Martin Sulzmann
}
\maketitle

\tableofcontents
\newpage

\section{Einleitung}
\subsection{Vorwort}

Ziel dieser Arbeit ist es, das Konstrukt der Futures und Promises zu untersuchen.
Dabei wird die Implementierung in der Programmiersprache Scala herangezogen.

Futures und Promises gehen zurück auf die Sprachen 
\emph{Mozart Programming System}, \emph{Alice} und \emph{MultiLisp}. Sie sind
heute in einer Vielzahl von Sprachen verfügbar, oftmals durch drittanbieter
Bibliotheken.

%TODO http://www.ps.uni-saarland.de/alice/manual/futures.html
%TODO http://mozart.github.io/publications/abstracts/oz-futures.html

Der vollständige Code dieser Seminararbeit ist unter \cite{code} verfügbar.

\subsubsection{Scala}

Mit Scala habe ich mich bereits in meiner Projektarbeit \cite{Wilhelm13} befasst, darum soll es 
in dieser Arbeit bei einer kurzen Vorstellung der wichtigsten Merkmale dieser
Sprache bleiben.

Scala ist eine objektorientierte und funktionale Sprache für die JVM. Sie
leistet Pionierarbeit für das Java-Ökosystem, in dem sie Features implementiert
die von Java übernommen werden. Als Beispiel sei auf die Lambda-Ausdrücke in
Java 8 verwiesen.

Die Sprache findet auch in der Industrie Anwendung, so wird sie unter Anderem von Twitter,
Sony und Foursquare eingesetzt (vgl. \cite{scalaInEnterprise}).

Ein starker Fokus liegt darauf, nebenläufige Programmierung in der Multicore-Ära
zu ermöglichen. Erreicht wird dies durch Sprachkonstrukte wie \emph{Actors},
\emph{Futures} und \emph{Promises}.


\section{Futures und Promises}

\subsection{Historie}

Die erste Erwähnung von Futures geht zurück auf das Jahr 1977 
(Siehe \cite{Baker:1977:IGC:872734.806932}).

Implementierung in Mozart \cite{aliceFuture}

Implementierung in Alice \cite{futures:98}

Basis für Implementierung in Scala: \cite{Lea:2000:JFF:337449.337465}

\subsection{Implementierung in scala.concurrent}

Futures und Promises sind für Scala beschrieben in dem \gls{sip}-Dokument SIP-14 
(vgl. \cite{sip14}).

Die Imlementierung in Scala kann unter \cite{scalaConcurrentCode} abgerufen werden.
%TODO https://github.com/scala/scala/blob/master/src/library/scala/concurrent/Future.scala
%TODO https://github.com/scala/scala/blob/master/src/library/scala/concurrent/Promise.scala

\section{Anwendungsfälle für Futures und Promises}

\subsection{Ein Urlaub in der Südsee}

\subsection{...}
\subsection{...}
\subsection{...}

\newpage
\section{Zusammenfassung und Fazit}

Lorem.

\newpage

\section{Quellen- und Literaturverzeichnis}

\bibliographystyle{plain}
\bibliography{seminararbeit_florian_wilhelm_36873}

\section{Abbildungs- und Tabellenverzeichnis}

\renewcommand{\listfigurename}{Verzeichnis der Abbildungen}
\listoffigures

\newpage

\printglossary

\end{document}
