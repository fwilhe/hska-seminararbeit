\section{Anwendungsfälle}

In diesem Kapitel werden verschiedene Anwendungsfälle geschildert.
Diese sollen der Illustration der Verwendung von \emph{Futures} und
\emph{Promises} dienen.

\subsection{Urlaubsplanung}

Dieser Anwendungsfall illustriert die Verwendung und Kombination
von \emph{Futures}.

\paragraph{Szenario}

Es wird ein Urlaub geplant. Dabei gibt es drei Dinge, die erledigt
werden müssen. Es muss der \emph{Wechselkurs} der Währung des
Reiseziels ermittelt werden. Wenn dies geschehen ist muss ein
\emph{Hotelzimmer} gebucht werden. Als letzter Schritt muss der
\emph{Flug} gebucht werden. Eine von den drei zuvor genannten
Schritten unabhängige Tätigkeit ist das Packen der Koffer. Während
also zum Beispiel auf die Antwort des Servers der Bank gewartet wird
kann an dieser Stelle sinnvolle Arbeit geschehen.

Für jeden dieser Schritte gibt es eine gewisse Wahrscheinlichkeit
dass er nicht erfolgreich ist. So kann es zum Beispiel passieren,
dass der Server zum Abfragen der Wechselkursen nicht online ist.

Ein weiteres Risiko im Umgang mit Fremdwährungen ist es, dass der
Wechselkurs sehr ungünstig ist. Für diesen Fall ist es eine Option
sich ein zweites Reiseziel offen zu halten. Es besteht zwar nach wie
vor die Gefahr dass beide Währungen nicht zu hinnehmbaren Kursen
erhältlich sind, aber dieser Fall soll hier nicht betrachtet werden.

Ebenso gibt es bei der Buchung von Hotelzimmer und Flug eine gewisse
Wahrscheinlichkeit eines technischen Fehlers.

Sollte einer dieser Fehlerzustände eintreten, so kann dies mit
\emph{Futures} modelliert werden in dem diese eine Exception
statt eines zu erwartenden Wertes enthalten.

\paragraph{Pseudocode} für diesen Anwendungsfall:

\lstinputlisting
    [caption={Code zum Anwendungsfall \glqq Urlaubsplanung\grqq{} },
       label = lst:holiday,
       captionpos=b]
 {../code/holiday/Holiday.pseudo}
 
\paragraph{Diskussion des Codes}

In diesem Programm werden zwei \emph{Futures} erzugt, die jeweils
den Wechselkurs für eine Währung enthalten. Mittels der
\texttt{fallbackTo}-Methode wird eine Währung ausgewählt.
Das \emph{rateDollar}-Objekt genießt Priorität, wird also ausgewählt
außer wenn es eine Exception enthält.

Das so erstellte Objekt \emph{selectedRate} beinhaltet nun entweder
den \emph{Future} mit dem Wechselkurs des US-Dollar oder mit dem des
Schweizer Franken.

Auf diesem Objekt wird die \texttt{map}-Methode aufgerufen. Diese wird,
wenn der Wechselkurs akzeptabel ist das Buchen des Hotelzimmers beauftragen,
wenn nicht eine Exception werfen.

Zuletzt wird ein \emph{Future} erstellt um den Flug zu buchen. Dies
geschiet in Abhänigkeit vom Erfolg des Buchens eines Hotelzimmers.

Wenn auch dieser \emph{Future} erfolgreich abschließt, dann ist die
Liste der drei Aufgaben abgearbeitet.

Zuletzt steht der Methodenaufruf \texttt{packSuitcase()}, der blockierend
im Main-Thread läuft. Da die Methodenaufrufe der \emph{Futures} alle
nicht blockierend sind laufen diese Aufgaben nebenläufig.

\subsection{Bauarbeiten}

Dieser Anwendungsfall demonstriert die Verwendung von \emph{Promises}.

\paragraph{Szenario}

Es finden Bauarbeiten statt. Beteiligt sind zwei handelnde Personen,
ein Meister und sein Lehrling. Als diese mit der Arbeit beginnen möchten
fällt Ihnen auf, dass sie den Werkzeugkasten vergessen haben. Der
Lehrling bricht auf um diesen zu besorgen. Die Zwischenzeit kann
der Meister nutzen um Messungen vorzunehmen. Wenn der Lehrling mit
dem Werkzeugkasten zurückkommt kann er sich einer anderen Aufgabe
(dem säubern der Baustelle) widmen, während der Meister das für die
anstehende Arbeit passende Werkzeug auswählen kann.

Würden hier blockierende Methodenaufrufe eingesetzt, dann wäre es nicht
möglich die Zeit in der die Werkzeugkiste geholt wird sinnvoll zu
nutzen.

\paragraph{Pseudocode} für diesen Anwendungsfall:

\lstinputlisting
    [caption={Code zum Anwendungsfall \glqq Bauarbeiten\grqq{} },
       label = lst:constructionsite,
       captionpos=b]
 {../code/constructionsite/Constructionsite.pseudo}

\paragraph{Diskussion des Codes}

Es wird ein \emph{Promise} erzeugt und der dazugehörige \emph{Future}
gespeichert. Der Lehrling und der Meister werden als \emph{Future}
modelliert. Der Future des Lehrlings wird mit dem Methodenaufruf
\texttt{getToolbox()} blockiert, da er währenddessen keine weitere
Tätigkeit ausführen kann. Der Future des Meisters ist davon nicht
betroffen. Sobald der Lehrling das Versprechen einlöst wird die
\texttt{onSuccess}-Callback-Methode des zugehörigen Futures ausgelöst.
An dieser Stelle kann der Meister seine Vorarbeiten beenden und sich
an die eigentliche Arbeit machen.

\paragraph{Mehrmaliges Schreiben} ist bei Promises nicht erlaubt.
Nehmen wir an, der Code wird folgendermaßen verändert:

\lstinputlisting
    [caption={Modifizierter Code zum Anwendungsfall \glqq Bauarbeiten\grqq{} },
       label = lst:constructionsite_invalid,
       captionpos=b]
 {../code/constructionsite/Constructionsite_invalid.pseudo}
 
Der zweite Aufruf der \texttt{success()}-Methode des \emph{Promises}
wird eine Exception auslösen, unabhängig vom Parameter.

Dieses Verhalten ist gewollt und macht \emph{Promises} berechenbar.
Es gibt zwei Möglichkeiten wie diese abgeschlossen werden könnnen:
Entweder durch einen Wert von dem Typ mit dem sie typisiert sind,
oder durch eine Exception falls bei der Berechnung ein Fehler auftrat.
Ein Beispiel in unserem Anwendungsfall dafür könnte sein, dass der
Lehrling den Werkzeugkasten im Auto nicht finden kann.
