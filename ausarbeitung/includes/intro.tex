\section{Einleitung}

Ziel dieser Seminararbeit ist es, die Sprachmittel \emph{Future}
und \emph{Promise} am Beispiel der Programmiersprache \emph{Scala}
zu untersuchen. Anhand von Anwendungsbeispielen wird verdeutlicht, 
in welchen Situationen diese Sprachmittel hilfreich sind.

Der Code in diesem Dokument ist als Pseudocode zu verstehen, der für die
Sache unwichtige Details abstrahiert.

Die Programmbeispiele (siehe \cite{code}) wurden mit
\emph{Scala} in Version 2.10.3 geschrieben und getestet. Diese bietet
eine eigene Implementierung von \emph{Futures} und \emph{Promises}.
Eine kurze Anleitung zur Verwendung dieser Implementierung findet sich im
\emph{Scala Improvement Process}-Dokument 14 \cite{sip14}
und in der Dokumentation zur Programmiersprache \cite{scalaDokuFP}.

\emph{Futures} und \emph{Promises} sind mit dem Ziel entworfen worden,
die nebenläufige Programmierung zu vereinfachen. Nebenläufige Programmierung
war bereits vor vielen Jahren sinnvoll, um den Prozessor besser auszulasten,
aber seit Mehrkernsysteme Einzug in unseren Alltag gehalten haben
ist sie noch wichtiger geworden als zuvor.

Dabei bietet \emph{Scala} mehr Abstraktionsmöglichkeiten um die Arbeit
mit nebenläufigen Problemstellungen zu vereinfachen. So hat es zum
Beispiel das \emph{Aktorenmodell} aus der funktionalen Programmiersprache
\emph{Erlang} übernommen.

Beide dieser Ansätze bieten unterschiedliche Herangehensweisen und
Vor- und Nachteile. So erlauben \emph{Aktoren} es, einen internen Zustand
zu speichern. \emph{Futures} dagegen sind einfach kombinierbar.

Der Einsatzzweck von \emph{Futures} liegt vor allem darin, bestimmte
aufwändige Berechnungen einmalig durchführen zu lassen.

Dass es sich bei \emph{Futures} und \emph{Promises} um keine neuen
Ideen handelt wird deutlich, wenn man \cite{Baker:1977:IGC:872734.806932}
betrachtet. Bereits im Jahre 1977 wird die grundlegende Idee beschrieben.

\emph{Futures} werden als Sprachkonstrukte beschrieben, die asynchron laufen. Sie 
könnten zum Beispiel in der Form 
\begin{lstlisting}
(ENTWEDER <ausdruck_1> <ausdruck_2> .. <ausdruck_n>)
\end{lstlisting}
angegeben werden. Diese Notation sollte alle n Ausdrücke nebenläufig ausführen, 
die Rückgabe sollte die des ersten Ausdrucks sein, der ein Ergebnis zurückliefert.
Die Idee war, jeden Ausdruck auf einem eigenen Prozessor auszuführen, 
wenigstens aber alle Ausdrücke auf mehrere Prozessoren zu verteilen. Hier ist in 
einem 1977 veröffentlichten Papier bereits eine zentrale Problemstellung der 
heutigen Zeit beschrieben: Die effiziente Softwareentwicklung in der Multicore-Ära.

Anzumerken ist, dass nebenläufig nicht gleichbedeutend ist mit parallel.
Nebenläufige Programme können auch auf Einkerncomputern ausgeführt werden.
Ein positiver Effekt der nebenläufigen Programmierung ist, dass ein 
Programm ohne Probleme parallel laufen kann; dies ist jedoch nicht ihr 
alleiniges Ziel.

