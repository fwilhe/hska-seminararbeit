\section{Zusammenfassung und Fazit}

\subsection{Zusammenfassung}

Die vorliegende Seminararbeit führt in das Arbeiten mit \emph{Futures}
und \emph{Promises} ein. Sie motiviert deren Verwendung anhand der
Eigenschaft nicht blockierend zu sein und arbeitet heraus wie sie
zu verwenden sind. Dabei wird insbesondere auf die Möglichkeit
zur Kombination mehrerer Objekte eingegangen.

Anhand von aus dem echten Leben gegriffenen Anwendungsfällen wird
verdeutlicht in welchen Situationen \emph{Futures} und \emph{Promises}
hilfreich sein können.

\subsection{Fazit}

Ich persönlich habe mich durch diese Arbeit eingehend mit 
Abstraktionsschichten die der nebenläufigen Programmierung dienen
beschäftigen können. Die Bedeutung der Nebenläufigkeit darf nicht
unterschätzt werden. Besonders in der heutigen Welt, in der zunehmend
auch kleine Computer mehr als einen Rechenkern zur Verfügung haben
muss zur Kentnis genommen werden, dass die Welt nicht sequenziell
arbeitet. 

\subsection{Ausführen der Code-Beispiele}

Alle Programme, die in dieser Arbeit diskutiert wurden sind 
vollständig in dem GitHub-Repository \cite{code} veröffentlicht.
Um sie auszuführen ist eine Installation von \emph{Scala} in der
Version 2.10 oder höher notwendig.

Die Beispiele enthalten alle ein eigenes \emph{Makefile}, welches
mit dem Defaulttarget \emph{build} die Quellen übersetzt und mit
dem target \emph{run} das Programm ausführt.

Alternativ kann auch die Kommandozeile \\
\texttt{scalac *.scala \&\& scala de.hska.wifl1011.seminararbeit.Main} \\
von Hand ausgeführt werden.

\subsection{Blick über den Tellerrand}

\textbf{TODO: DIESES KAPITEL ENTWEDER AUSBAUEN ODER LÖSCHEN!}

Viele Programmiersprachen bieten mittlerweile Implementierungen von
\emph{Futures} und zum Teil von \emph{Promises}.

So befindet sich zum Beispiel seit \emph{C++11} \emph{std::future} und
\emph{std::future} im \emph{C++}-Standard (vgl.: \cite{cpp11FAQ}).

\paragraph{\emph{Futures} und \emph{Promises} in \emph{Scala}}

Es gab bereits seit längerem verschiedene Implementierungen dieser Sprachmittel,
unter anderem im Akka-Framework, in Skalaz, in den Twitter-Util-Klassen so wie
in java.util.concurrent (vgl. \cite{futuresTry}).

Um der Fragmentierung im Umfeld der Programmiersprache entgegenzuwirken und
diese Techniken für alle Scala-Entwickler zugänglich zu machen, wurde auf Basis
des SIP-Dokuments SIP-14 eine Implementierung im scala.concurrent-Package
vorgenommen.

\subsection{Echte Anwendungen von \emph{Futures} und \emph{Promises}}

\textbf{TODO: DIESES KAPITEL ENTWEDER AUSBAUEN ODER LÖSCHEN!}

Ein Beispiel für den echten Einsatz von \emph{Futures} und \emph{Promises}
ist das \emph{play framework} (vgl.: \cite{playframework}). Es wird
von \emph{Typesafe} entwickelt, der Firma die auch hinter \emph{Scala}
steht.

Eine einfache Suche nach dem Schlüsselwort \glqq Future\grqq{} fördert
im Quelltext des Frameworks knapp 1000 Treffer zu Tage.
