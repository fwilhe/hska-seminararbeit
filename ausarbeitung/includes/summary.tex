\section{Zusammenfassung und Fazit}

\subsection{Zusammenfassung}

Die vorliegende Seminararbeit führt in das Arbeiten mit \emph{Futures}
und \emph{Promises} ein. Sie motiviert deren Verwendung anhand der
Eigenschaft, nicht blockierend zu sein und arbeitet heraus wie sie
zu verwenden sind. Dabei wird insbesondere auf die Möglichkeit
zur Kombination mehrerer Objekte eingegangen.

Anhand von aus dem echten Leben gegriffenen Anwendungsbeispielen wird
verdeutlicht, in welchen Situationen \emph{Futures} und \emph{Promises}
hilfreich sein können.

\subsection{Fazit}

Ich persönlich habe mich durch diese Arbeit eingehend mit 
Abstraktionsschichten, die der nebenläufigen Programmierung dienen,
beschäftigen können. Die Bedeutung der Nebenläufigkeit darf nicht
unterschätzt werden. Besonders in der heutigen Welt, in der zunehmend
auch kleine Computer mehr als einen Rechenkern zur Verfügung haben
muss zur Kenntnis genommen werden, dass die Welt nicht sequenziell
arbeitet. 

\subsection{Ausführen der Code-Beispiele}

Alle Programme, die in dieser Arbeit diskutiert wurden sind 
vollständig in dem GitHub-Repository \cite{code} veröffentlicht.
Zur Ausführung ist eine Installation von \emph{Scala} in der
Version 2.10 oder höher notwendig.

Die Beispiele enthalten alle ein eigenes \emph{Makefile}, welches
mit dem Defaulttarget \emph{build} die Quellen übersetzt und mit
dem target \emph{run} das Programm ausführt.

Alternativ kann auch die Kommandozeile \\
\texttt{scalac *.scala \&\& scala de.hska.wifl1011.seminararbeit.Main} \\
von Hand ausgeführt werden.



