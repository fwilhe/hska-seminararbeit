\section{\emph{Futures} in anderen Sprachen}

Viele Programmiersprachen bieten mittlerweile Implementierungen von
\emph{Futures} und zum Teil von \emph{Promises}.

\paragraph{Java} bietet seit Version 1.5 der Sprache eine 
\texttt{java.util.concurrent.Future}-Klasse an (vgl.: \cite{javadocfuture}).

Der Zugriff auf das Ergebnis eines \emph{Futures} funktioniert über
eine \texttt{get()}-Methode, die den Aufrufer blockiert bis das
Ergebnis vorliegt.

Ein Nachteil dieser Implementierung besteht in den fehlenden
\emph{Callback}-Methoden. So ist es nicht möglich, informiert
zu werden wenn ein \emph{Future} ein Ergebnis enthält, stattdessen
ist es notwendig \texttt{get()} aufzurufen und im Fehlerfall
eine Exception zu behandeln.

Es ist zwar möglich eine Callback-Klasse auf Basis des \emph{Interface}
\texttt{Runnable} zu erstellen, doch dies ist aus Entwicklersicht
nicht mit dem Komfort vergleichbar, den \emph{Scala Futures} bieten.

Ein weiterer Nachteil besteht darin dass die aus \emph{Scala} bekannten
Kombinatoren nicht verfügbar sind. Es ist zwar möglich deren Funktion
nachzuprogrammieren, doch dies sorgt für erheblichen Mehraufwand
auf Seiten des Anwendungsentwicklers und macht \emph{Java Futures}
damit nicht besonders attraktiv.

Drittanbieter wie zum Beispiel die \emph{Guava Google Libraries for Java}
bieten eigene Implementierungen, die einzelne Nachteile ausbessern.
So gibt es dort zum Beispiel einen \emph{ListenableFuture} der die
bekannten \texttt{onSuccess()} und \texttt{onFaulure()}-Callback-Methoden
kennt. Zusätzlich bietet diese Klasse Methoden die zum Teil den
Kombinatoren aus \emph{Scala} ähneln.

\paragraph{C++} Seit \emph{C++11} kennt der \emph{C++}-Standard
\emph{std::future} und \emph{std::promise}(vgl.: \cite{cpp11FAQ}).

\emph{Futures} in \emph{C++} funktionieren nicht gleich wie in
\emph{Scala}, so sind zum Beispiel die Kombinatoren wie in Java auch 
hier nicht Verfügbar.

Der Zugriff auf den Wert eines \emph{Futures} erfolgt ebenfalls über die
\texttt{get()}-Methode, die den Aufrufer so lange blockiert bis
das Ergebnis des \emph{Futures} vorliegt.

\paragraph{Scala} Es gab bereits seit längerem verschiedene 
Implementierungen dieser Sprachmittel,
unter anderem im Akka-Framework, in Skalaz, in den Twitter-Util-Klassen sowie
in java.util.concurrent (vgl. \cite{futuresTry}).

Um der Fragmentierung im Umfeld der Programmiersprache entgegenzuwirken und
diese Techniken für alle Scala-Entwickler zugänglich zu machen, wurde auf Basis
des SIP-Dokuments SIP-14 eine Implementierung im scala.concurrent-Package
vorgenommen.
