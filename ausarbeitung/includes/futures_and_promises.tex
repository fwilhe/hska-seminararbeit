\section{Futures und Promises}

\subsection{Motivation}

Methodenaufrufe in üblichen Programmiersprachen sind blockierend. Das bedeutet,
dass das Hauptprogramm so lange blockiert wird, wie der Methodenaufruf läuft.
So wird in Listing \ref{lst:codeBlocking} die Methode \texttt{tuEtwas()} eines Objektes aufgerufen.

\begin{lstlisting}[caption={Blockierender Methodenaufruf},label={lst:codeBlocking},captionpos=b]
main() {
  objekt.tuEtwas()
  tuEtwasAnderes()
}
\end{lstlisting}

Führen wir das Schlüsselwort \texttt{nebenlaeufig} in unseren Pseudocode ein,
so können wir ausdrücken, dass die aufgerufene Methode nicht blockieren soll,
sondern dass im Hauptthread weitergearbeitet werden kann.

\begin{lstlisting}[caption={Nebenläufiger Methodenaufruf},label={lst:codeConcurrent},captionpos=b]
main() {
  nebenlaeufig objekt.tuEtwas()
  tuEtwasAnderes()
}
\end{lstlisting}

Nun kann \texttt{tuEtwasAnderes()} \emph{nebenläufig} ausgeführt werden. Damit
ist es bei mehreren Prozessoren möglich, den Code parallel auszuführen.

Eine Analogie aus dem IT-Umfeld sind Unix-Shells. Wenn zum Beispiel ein
\texttt{find}-Befehl ausgeführt wird, so blockiert die Shell bis der Befehl
abgeschlossen ist. Durch Anhängen eines kaufmännischen Undzeichens ist es möglich,
den Befehl im Hintergrund laufen zu lassen, wodurch es möglich wird im selben
Terminal weiter zu arbeiten während der Befehl im Hintergrund (also nebenläufig)
läuft.

\subsection{Futures}

Im SIP-14 wird ein \emph{Future} wie folgt beschrieben:
\begin{quote}
Futures provide a nice way to reason about performing many operations in 
parallel– in an efficient and non-blocking way. The idea is simple, a Future 
is a sort of placeholder object that you can create for a result that doesn’t 
yet exist. Generally, the result of the Future is computed concurrently and can 
be later collected. Composing concurrent tasks in this way tends to result in 
faster, asynchronous, non-blocking parallel code.

A Future object either holds a result of a computation or an 
exception in the case that the computation failed.
\end{quote}

Das erklärte Ziel ist es, einen angenehmen Weg zu schaffen um nicht
blockierend (\glqq non-blocking\grqq{}) zu programmieren.

Alles was innerhalb eines \emph{Futures} passiert läuft nicht im
Hauptthread.

\paragraph{Einfaches Beispiel in Pseudocode}

\lstinputlisting
    [caption={Pseudocode zum Minimalbeispiel eines Futures },
       label = lst:scala_mini_future,
       captionpos=b]
 {../code/minimal/future/Future.pseudo}
 
In Listing \ref{lst:scala_mini_future} ist der Umgang mit einem \emph{Future}
skizziert. Sie sind generische Typen, es ist also Möglich \emph{Futures}
zu verwenden die beliebige Typen beinhalten. Der Code des \emph{Futures}
wird innerhalb des \texttt{future \{ ... \}}-Blocks definiert. Der
Rückgabewert muss dem Typen entsprechen, den der \emph{Future} zurückgibt.

\subsection{Promises}

Ein \emph{Promise} hingegen wird im SIP-14 folgendermaßen beschrieben:
\begin{quote}
A promise can be thought of as a writeable, single-assignment container, which 
completes a future. That is, a promise can be used to successfully complete a 
future with a value (by \glqq completing\grqq{} the promise) using the \texttt{success} method. 
Conversely, a promise can also be used to complete a future with an exception, 
by failing the promise, using the \texttt{failure} method.
\end{quote}

Das Konzept \emph{Promise} baut auf den \emph{Futures} auf. Es bietet
einmaligen Schreibzugriff auf ein Objekt, das zum gegebenen Zeitpunkt
noch nicht beschrieben werden kann, zum Beispiel weil dieser Wert
noch nicht berechnet worden ist.

\paragraph{Einfaches Beispiel in Pseudocode}

\lstinputlisting
    [caption={Pseudocode zum Minimalbeispiel eines Promises },
       label = lst:scala_mini_promise,
       captionpos=b]
 {../code/minimal/promise/Promise.pseudo}
 
Im Beispiel in Listing \ref{lst:scala_mini_promise} ist ein \emph{Promise} zu sehen.
Jedes \emph{Promise} beinhaltet ein \emph{Future}. Im Programm kann
nun so lange gerechnet werden, bis der Wert mit dem das \emph{Promise}
vervollständigt werden kann verfügbar ist. Im Beispiel ist dieser
Wert der Einfachheit halber hart codiert, in der Praxis wird er das
nicht sein.

\paragraph{Vergleich} von \emph{Futures} und \emph{Promises}.

\begin{table}[h]
\begin{tabular}{lllll}
 & \textbf{Eigenschaft} & \textbf{Future} & \textbf{Promise} &  \\
 & Lesbar & Ja & Nein &  \\
 & Schreibbar & Nein & Ja &  \\
 & Beinhaltet anderen & Nein & Ja &  \\
 &  &  &  &  \\
 &  &  &  &  \\
 &  &  &  & 
\end{tabular}
\end{table}

\subsection{Kombinatoren}

Kombinatoren werden beschrieben in der offiziellen \emph{Scala}-Dokumentation (vgl.: \cite{scalaDokuFP}).
Sie erlauben es, mehrere Futures zu verbinden, ohne die \texttt{onComplete()}-Methoden
zu verschachteln. Dadruch können komplexe Zusammenhänge relativ einfach
im Code formuliert werden.

\paragraph{map} ist einer der grundlegenden Kombinatoren. Er bildet
einen \emph{Future} nach bestimmten Regel auf einen anderen ab.

In \ref{lst:scala_map} ist ein einfaches Beispiel dafür gegeben.
In diesem Beispiel wird eine Gurke geerntet. Diese besitzt als
Eigenschaft eine Krümmung (in Grad). Da sich nur einigermaßen gerade
Gurken gut verkaufen lassen wird diese Future per \texttt{map}-Funktion
auf einen anderen Future abgebildet. Hier findet die Prüfung statt ob
die Gruke verkehrsfähig ist oder weggeworfen werden muss.

Falls sie nicht gerade genug ist dann enthält der \texttt{harvestFuture}
eine Exception.

\lstinputlisting
    [caption={Pseudocode zum \emph{map}-Kombinator },
       label = lst:scala_map,
       captionpos=b]
 {../code/harvestluna/Harvest.pseudo}
 
\paragraph{fallbackTo} ist ein Kombinator der zwei \emph{Futures}
verbindet. Dabei wird überprüft ob der erste einen Wert (und damit
keine Exception) enthält. Wenn dies der Fall ist wird dieser ausgewählt.
Falls der erste eine Exception enthält wird der zweite überprüft. Enthält
dieser einen Wert, dann wird er ausgewählt. Enthält auch der zweite
eine Exception, dann enthält auch der neu erstellte eine Exception.

\lstinputlisting
    [caption={Pseudocode zum \emph{fallbackTo}-Kombinator },
       label = lst:scala_fallbackTo,
       captionpos=b]
 {../code/lookingforsomething/Looking.pseudo}
 
In \ref{lst:scala_fallbackTo} ist ein einfaches Beispiel gegeben:
Es gibt eine Suchanfrage und verschiedene Anbieter bei denen diese
Anfrage bearbeitet werden kann. Sollte der eine Anbieter ausfallen,
dann steht immer noch das Ergebnis des anderen Anbieters zur Verfügung.
Im worst-case stehen beide Anbieter nicht zur Verfügung, in diesem
Fall ist es nicht möglich ein Ergebnis anzuzeigen.

\paragraph{firstCompletedOf} wählt den \emph{Future} aus, der als
erstes beendet ist. Dies ist nützlich wenn es um Geschwindigkeit
geht. Wenn zum Beispiel ein Notfall passiert und über verschiedene
Kanäle ein Notruf abgesetzt wird, dann ist das einzig sinnvolle Kriterium
zur Auswahl über welchen Kanal als erstes Hilfe zu erwarten ist.

\lstinputlisting
    [caption={Pseudocode zum \emph{firstCompletedOf}-Kombinator },
       label = lst:scala_firstCompletedOf,
       captionpos=b]
 {../code/accident/Accident.pseudo}
 
Listing \ref{lst:scala_firstCompletedOf} skizziert diesen
Anwendungsfall.

\paragraph{For-Comprehension}
