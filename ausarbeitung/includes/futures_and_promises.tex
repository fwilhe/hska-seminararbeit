\section{Futures und Promises}

\subsection{Futures}

Im SIP-14 wird ein \emph{Future} wie folgt beschrieben:
\begin{quote}
Futures provide a nice way to reason about performing many operations in 
parallel– in an efficient and non-blocking way. The idea is simple, a Future 
is a sort of placeholder object that you can create for a result that doesn’t 
yet exist. Generally, the result of the Future is computed concurrently and can 
be later collected. Composing concurrent tasks in this way tends to result in 
faster, asynchronous, non-blocking parallel code.

A Future object either holds a result of a computation or an 
exception in the case that the computation failed.
\end{quote}

% non blocking



\subsection{Promises}

Ein \emph{Promise} hingegen wird im SIP-14 folgendermaßen beschrieben:
\begin{quote}
A promise can be thought of as a writeable, single-assignment container, which 
completes a future. That is, a promise can be used to successfully complete a 
future with a value (by \glqq completing\grqq{} the promise) using the \texttt{success} method. 
Conversely, a promise can also be used to complete a future with an exception, 
by failing the promise, using the \texttt{failure} method.
\end{quote}


\paragraph{Vergleich} von ..
