%
% seminararbeit_florian_wilhelm_36873 Florian Wilhelm
% Erstellt am 19.03.2014
%
% Bauen mit dem beiliegenden Makefile

\documentclass[11pt,a4paper,titlepage,ngerman]{scrartcl}

% Deutsch mit neuer Rechtschreibung
\usepackage[ngerman]{babel}

% Umlaute direkt eingeben http://www.jr-x.de/publikationen/latex/tipps/besonderheiten.html
% Hatte hier ursprünglich "latin1" als Option, das ging aber nicht. Hängt wohl von der Codierung des Dokumentes ab.
\usepackage[utf8]{inputenc} 

% Einbinden von Grafiken
\usepackage{graphicx}

\usepackage{listings}

% Setzen der Eigenschaften der erzeugten PDF-Datei
\usepackage[pdftex,
    pdfauthor={Florian Wilhelm},
    pdfsubject={Seminararbeit Florian Wilhelm MatrNr 36873},
    pdftitle={Futures and Promises in Scala},    
    pdfproducer={Latex with hyperref},
    pdfcreator={pdflatex}]{hyperref}

% Glossar-Paket laden und Glossar erzeugen
% Hierfür sind mehrere Übersetzungsvorgänge nötig. Dafür gibt es ein Build-Target im Makefile.
\usepackage[toc]{glossaries}
\makeglossaries

%
% Glossar
%
\newglossaryentry{haskell}{name={Haskell}, description={ist eine rein funktionale Programmiersprache. Sie ist unter \url{http://www.haskell.org} als freie Software verfügbar}}

% Template: \newglossaryentry{yyy}{name=xxx, description={}} 

%
% Glossar Ende
%

%Römische Zahlen
% http://www.mrunix.de/forums/showthread.php?t=44246
\newcommand{\RM}[1]{\MakeUppercase{\romannumeral #1{.}}}

\begin{document}

% Zeilenumbrüche in Code-listings
\lstset{breaklines=true}

\titlehead{
	\includegraphics[width=0.9\linewidth]{pic/500px-Hska_logo.png}
}
\title{Seminararbeit}
\subtitle{Futures and Promises in Scala}
\author{Florian Wilhelm \\
		Matrikelnummer: 36873\\
		Wintersemester 2013/2014}
\publishers{
    \textbf{Betreuer:} Prof. Dr. Martin Sulzmann
}
\maketitle

\tableofcontents
\newpage

\section{Einleitung}


\subsection{Vorwort}

Lorem.

\newpage
\section{Zusammenfassung und Fazit}

Lorem. \cite{scalalang}

\newpage

\section{Quellen- und Literaturverzeichnis}

\bibliographystyle{plain}
\bibliography{seminararbeit_florian_wilhelm_36873}

\section{Abbildungs- und Tabellenverzeichnis}

\renewcommand{\listfigurename}{Verzeichnis der Abbildungen}
\listoffigures

\newpage

\printglossary

\end{document}
