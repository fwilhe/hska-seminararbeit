\documentclass{beamer}
\usepackage[ngerman]{babel}
\usepackage[utf8]{inputenc}
\usepackage{listings}

%\usetheme{Ilmenau}

%remove navigation symbols
%http://stackoverflow.com/questions/3017030/hiding-the-presentation-controls-in-latex-beamer-presentation
\setbeamertemplate{navigation symbols}{}

\title[]{Seminararbeit\\Futures and Promises in Scala}
\author{Florian Wilhelm}
\date{\today{}}

\begin{document}

% Titelseite
\frame{
\titlepage
}

% Inhaltsverzeichnis
\frame{
\frametitle{Inhaltsverzeichnis}
\tableofcontents
}

\section{Einleitung}

\setcounter{subsection}{1}
\begin{frame}
  \frametitle{Nebenläufigkeit}
  \begin{itemize}
    \item{Klassisch: Sequentiell}
    \item{Problem: CPUs werden nicht schneller, aber mehr}
    \item{Lösung: Nebenläufigkeit}
    \item{Problem: Nicht leicht mit c(++)/java/...}
    \item{Lösung: Actors (Erlang, Scala), Futures/Promises, Channels (Golang)}
   \end{itemize}
\end{frame}

\section{Future}

\setcounter{subsection}{1}
\begin{frame}
  \frametitle{Future}
   \begin{itemize}
    \item{Erste Beschreibung: 1977}
    \item{Implementiert seit Ende der 1990er}
    \item{Zugriff auf nebenläufige Berechung}
    \item{Beliebig oft lesbar}
    \item{Aufruf blockiert nicht}
    \item{Kombinierbar}
   \end{itemize}
\end{frame}

\setcounter{subsection}{1}
\begin{frame}
  \frametitle{Future}
    \lstinputlisting
        [caption={Pseudocode zum Minimalbeispiel eines Futures },
           label = lst:scala_mini_future,
           captionpos=b]
     {../code/minimal/future/Future_nocomment.pseudo}
\end{frame}

\setcounter{subsection}{1}
\begin{frame}
  \frametitle{Kombinatoren}
   \begin{itemize}
    \item{Funktionen die auf Futures anwendbar sind}
    \item{Elegante Möglichkeit Sachverhalte auszudrücken}
    \item{...}
   \end{itemize}  
\end{frame}

\setcounter{subsection}{1}
\begin{frame}
  \frametitle{fallbackTo}
    \lstinputlisting
        [caption={Pseudocode zum fallbackTo-Kombinator },
           label = lst:scala_fallbackTo,
           captionpos=b]
     {../code/lookingforsomething/Looking.pseudo}
\end{frame}


\section{Promise}

\setcounter{subsection}{1}
\begin{frame}
  \frametitle{Promise}
   \begin{itemize}
    \item{Gegenstück zum Future}
    \item{Einmalig beschreibbar}
    \item{...}
   \end{itemize}
\end{frame}

\setcounter{subsection}{1}
\begin{frame}
  \frametitle{Promise}
    \lstinputlisting
        [caption={Pseudocode zum Minimalbeispiel eines Promises },
           label = lst:scala_mini_promise,
           captionpos=b]
     {../code/minimal/promise/Promise_nocomment.pseudo}
\end{frame}

\section{Anwendungen}

\setcounter{subsection}{1}
\begin{frame}
  \frametitle{Use-Case 1: Urlaub}
   \begin{itemize}
    \item{Wechselkurs ermitteln}
    \item{Hotelzimmer buchen}
    \item{Flug buchen}
    \item{Koffer packen}
   \end{itemize}
\end{frame}

\setcounter{subsection}{1}
\begin{frame}
  \frametitle{Use-Case 2: Baustelle}
   \begin{itemize}
    \item{Werkzeugkoffer fehlt}
    \item{Arbeiten für die kein Werkzeug notwendig ist}
    \item{Werkzeugkoffer besorgen}
   \end{itemize}
\end{frame}

\section{}
\begin{frame}
  Vielen Dank für Ihre Aufmerksamkeit
\end{frame}

\end{document}
